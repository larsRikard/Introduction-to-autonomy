%% sample template file for a MSc Thesis
%% The default is with two sided setup:
\documentclass[%
oneside,    %% uncomment for onesided layout
project,    %% uncomment not thesis but project report
nosummary   %% uncomment if no summary page should be generated
]{USN-MSc}

% The following command removes the chapter names form the header
% (comment/remove) if you prefer to have them:
\pagestyle{plain}

% --- Bibliography setup ---
%%% default is the "ieee" style
\usepackage[style=ieee, sorting=none]{biblatex}
%%% If you want to use "author-year" style
%%% where `\cite{Foo2011}` generates "Foo et al. (2011)"
%%% and   `\parencite{Foo2011}` generates "(Foo et al. 2011)"
%%% then comment the line above and use
%\usepackage[style=authoryear]{biblatex}
%%% or
%%% if you want to use "alphabetic" style then use
%%% where `cite[Foo2011]` generates "[Foo11]"
%%% then comment the line above and use
%\usepackage[style=alphabetic]{biblatex}
%%% instead.
%% load the bib file:
\addbibresource{sources.bib}

\usepackage{lipsum} % just for providing fill text used in this template
\usepackage{array} % for adjusting tables?

% --- general setup ---
%% Please fill in the following parameters:
\newcommand{\mytitle}{%
%% title:
yoyoyo
}

\newcommand{\mysubtitle}{%
%% master programme (for thesis only)
%% uncomment the appropriate one:
%%Electrical Power Engineering
%Energy and Environmental Technology
%Industrial IT and Automation
%Process Technology
}

\newcommand{\mykeywords}{%
%% keywords (for thesis only):
<keyword one, keyword two, \ldots>
}

\newcommand{\myauthor}{%
%% author(thesis) or group code (project):
223786 Lars Rikard Rådstoga
}

\newcommand{\myparticipants}{
%% group participants (for project only)
<First participant>\\
<Second participant>\\
<Third participant>\\
<Fourth participant>
}

\newcommand{\supervisor}{%
%% supervisor:
<Supervisor's Name>}

\begin{document}

% --- title page setup ---
\USNtitlepage%
%% Please provide the following information:
%% #1 optional figure (set to {} if not wanted)
{%
  {\normalsize <optional figure>}
   \includegraphics[draft,width=\textwidth]{USN_logo_en}}
%% #2 Project partner:
{<Project partner>}
%% #3 Summary:
{%
\lipsum[6-7]
}


%\chapter*{Preface}
%\label{ch:preface}
%\addcontentsline{toc}{chapter}{Preface}
%\lipsum[1-3]
%\bigskip
%Porsgrunn, \today

%\myauthor %% for thesis
%\myparticipants %% for project


%% table of contents
\tableofcontents
\addcontentsline{toc}{chapter}{\contentsname}

%\listoffigures % out-comment if unwanted
%\addcontentsline{toc}{section}{\listfigurename}

%\listoftables  % out-comment if unwanted
%\addcontentsline{toc}{section}{\listtablename}



\chapter{Definition of autonomous systems}
\label{ch:definition}
\lipsum[4]
\section{Definitions}
There are many degrees of definitions for the phrase "autonomous system", which vary in levels of abstraction. 
As to not repeat the dictionary definitions given in the lectures, I decided to firstly ask the first "AI chatbot" I could find on Google, see Figure \ref{fig:chat-bot}
\begin{figure}[!ht]
  \centering
  \includegraphics[width=0.8\textwidth]{AutonomousSystemChatBotDefinition}
  \caption{An AI chatbot's response to the request of defining the phrase "autonomous system" \cite{ChaiChat90:online}.}
  \label{fig:chat-bot}
\end{figure}
This is close to the exact same definition that can be found in the textbook "Introduction to AI Robotics":
"Function autonomously" indicates that the robot can operate, self-contained, under all reasonable conditions without requiring recourse to a human operator \cite{murphy2000introduction}.
A third, very technical definition was found in "Autonomous Systems - An Architectural Characterization" and can be paraphrased \cite{Sifakis}:
An autonomous system consists of both an agent and a system, the agent contains modules dealing with the fundamental aspects of autonomy.
These modules are perception, refection, goal management, planning and self-adaptation. How well these features are coordinated relates to how well an agent performs in pursuing its goal.
And the system relates to sensors and actuators which the agent has to use and operate to perceive its environment and calculate its control strategy.

\section{My definition}
\label{sec:myDefinition}
A definition of an autonomous system could be:
An intelligent system: must be equipped with sensors for perceiving its own and its objectives' positions; 
have the ability to classify objectives, paths and obstacles; 
and be able to calculate optimal paths and actions to maximize its chances of success;
in case of failure the system should independently try to correct the error.

\section{Explanation}
\label{sec:myExplanation}
My system is supposed to pick fruit and be an unmanned ground vehicle (UGV).
So, the definition fits well in the following sense:
It has to know where it is positioned, where obstacles are and where the path to its objectives (fruit) is.
It has to tell ripe fruit from unripe and rotten, it should also know what a clear path looks like.
It needs to be able to control its actuators in an orderly fashion to move between bushes, loading stations, etc. 
And it has to move arms and claws or such to pick fruit. All without damaging bushes, people, itself, other fruit and more.

\chapter{System Diagrams}
\label{ch:diagrams}
\lipsum[4]

\section{Nine window diagram}
\label{sec:nineWindow}
\lipsum[4]
\begin{table}[!ht]
  \caption{Nine window diagram}
   \centering
    \begin{tabular}{ | m{3cm} | m{3cm} | m{3cm} | m{5cm} |}
     \hline
     Perspective & Past & Present & Future \\ \hline
     Super system & Farmer or group of farmers. & Farmer or group of farmers. 
     & Small scale groups for urban farming or small farms. \\ \hline
     System & Cheap labor harvesting by hand. & Autonomous harvester. 
     & Autonomous multipurpose robot for pollinating, weeding, trimming, manuring and more. \\ \hline
     Sub-system & Propulsion, navigation. & Propulsion, navigation, classification, collection, logistics. 
     & Tool or mode change, learning, weather/climate regulated. \\ \hline
     \end{tabular}
 \end{table}

\section{A frikking cartoon}
\label{sec:cartoon}
\lipsum[4]

\section{Context diagram}
\label{sec:contextDiagram}
\lipsum[4]

\section{IDEF0 (of the main function)}
\label{sec:IDEF0}
\lipsum[4]

\chapter{Impact on society}
\label{ch:impact}
\lipsum{4}

Removing jobs, what should people do now?
Are we losing our humanity?



\chapter{Legal and ethical aspects}
\label{ch:legal}
\lipsum{100}

\chapter{Human interaction(s) of your autonomous system}
\label{ch:human}
\lipsum{100}

% A dummy command that causes all bibliographyentries to be displayed
% even though there were not cited in the document. Used for demonstration
% purposes only in this template file.
~\nocite{*}

\cleardoublepage

% The bibliography should be displayed here...
\printbibliography[heading=bibintoc]
% You rather like to call the bibliography "References"? Then use this instead:
%\printbibliography[heading=bibintoc, title={References}]


\appendix
%\renewcommand{\appendixname}{Paper} %% So we get 'Paper X' displayed instead


\chapter[Short Title of Paper A]{Title of Paper A (probably very long and therefore not good to have in the header)}
\label{paper-a}

\paragraph{Note}
Since some papers tend to have a rather long title it is good to provide the optional short title which then will be displayed in the table of contents and header instead of the long original title.
On the openening page of the chapter the orginal \emph{long} title will be displayed.\bigskip

\emph{Short descriptive text of paper follows here.}\bigskip

The paper itself needs to be included in the published form as PDF on the next pages.
This can be done using the \texttt{pdfpages} package by adding the command:

\begin{verbatim}
\includepdf{pages=-,openright}{Filename}
\end{verbatim}

You can omit the \texttt{.pdf} when specifying the \texttt{Filename}. Also you should include always include the option \texttt{openright} since it would look strange to have the paper starting at the back of the cover page.

There are more options like only adding specific pages:
\begin{verbatim}
\includepdf{pages=2-6,openright}{Filename.pdf}
\end{verbatim}

For more options see Appendix~\ref{paper-b} where the most important pages of the \texttt{pdfpages} manual were inlcuded using \texttt{pdfpages}.


%%% Command to include a PDF file directly including all pages:


\chapter[Short Title of Paper B]{Title of Paper B}
\label{paper-b}
Short descriptive text of paper follows here.

Here we included the first five pages of the \texttt{pdfpages} manual itself.

\includepdf[pages=1-5,openright]{fig/pdfpages}

\end{document}

%%% Local Variables:
%%% mode: latex
%%% TeX-master: t
%%% End:
