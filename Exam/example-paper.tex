%% sample file for Modelica 2021 Conference paper
%% Copyright  Modelica Association
%
% This work may be distributed and/or modified under the
% conditions of the LaTeX Project Public License, either version 1.3
% of this license or (at your option) any later version.
% The latest version of this license is in
%   http://www.latex-project.org/lppl.txt
% and version 1.3 or later is part of all distributions of LaTeX
% version 2005/12/01 or later.
%
% This work is 'maintained' on GitHub:
%   https://github.com/modelica-association/conference-templates
%
% The Current Maintainers are: @akloeckner, @dietmarw, @bernhard-thiele
% With additions by @casella, @sjoelund
%
% This work consists of all files in the GitHub repository except
% a) The files indicated by .gitignore files
% b) The GitHub management files .gitignore, *.md
%
% This class is created from the template for the Modelica 2021 conference

%%% Use the more modern biber and biblatex for Unicode and @online support
\documentclass{modelica}
\addbibresource{example-paper.bib}

\usepackage[final]{pdfpages}

\hypersetup{%
  pdftitle  = {Fruit-Harvesting Robots: A Multifaceted Technology with Implications for Society, Sustainability, and the Agriculture Industry}, %TODO
  pdfauthor = {8510},
  pdfsubject = {Introduction to Autonomy},
  pdfkeywords = {Robotics, Automation, Autonomy, Agriculture},
  colorlinks,
  linkcolor=black,
  urlcolor=black,
  citecolor=black,
  pdfpagelayout = SinglePage,
  pdfcreator = pdflatex,
  pdfproducer = pdflatex}

% begin the document
\begin{document}
\thispagestyle{empty}

\title{Working Title - Fruit Picker Robot} %TODO
\author[1]{8510}
\affil[1]{IMS, Vestfold, Norway, {\small\texttt{223786@usn.no}}}
%\affil[2]{Company, Country, {\small\texttt{name3@company}}}

% \title{\textbf{Int. Modelica Conf. 2021 Paper Title}}
% \author{{\large
% Author Name$^1$ \quad Author Name$^1$ \quad Author Name$^2$\vspace{2mm}\\
%   {}$^1$Department, University, Country, \textsf{\{name1,name2\}@university.org}\\
%   {}$^2$Company, Country, \textsf{name3@company}}

\maketitle\thispagestyle{empty} %% <-- you need this for the first page
\abstract{%
This research paper examines the social, sustainable, technical, and applied aspects of autonomous fruit-harvesting robots, a technology with the potential to transform the agriculture industry. The study explores the impacts of these robots on labor shortages, the environment, and the agriculture industry, as well as the technical challenges and advances in their development. The applied aspects of autonomous fruit-harvesting robots, including their design and deployment, are also examined.
} %TODO

\noindent\emph{Keywords: Robotics, Automation, Autonomy, Agriculture}

\section{Introduction}
The development of autonomous fruit-harvesting robots has the potential to revolutionize the way that fruit is grown and harvested, with numerous social, sustainable, technical, and applied implications. As the demand for fresh fruit continues to grow, the use of these robots offers a promising solution to the challenges of labor shortages, increased costs, and environmental impacts in the agriculture industry.

This research paper aims to explore the various aspects of autonomous fruit-harvesting robots, including their social, sustainable, technical, and applied dimensions. By examining the current state of the technology and its potential future developments, we seek to shed light on the ways in which these robots could transform the agriculture industry and the potential impacts they may have on society and the environment.

In the social realm, autonomous fruit-harvesting robots have the potential to address labor shortages and reduce the reliance on human labor in the agriculture sector. They could also have implications for the distribution of wealth and the structure of rural communities, as well as for the safety and health of workers.

Sustainability is also a key concern in the development and use of autonomous fruit-harvesting robots. These robots have the potential to reduce the environmental impacts of fruit production, including greenhouse gas emissions, water use, and soil degradation. They could also contribute to food security and reduce waste by improving the efficiency and accuracy of fruit harvesting.

Technically, the development of autonomous fruit-harvesting robots requires utilization of a wide range of fields, including: robotics, computer vision, machine learning, and sensor technology. These tools are crucial to the success and widespread adoption of the technology.

Finally, the applied aspects of autonomous fruit-harvesting robots are vast and varied, ranging from the design and deployment of the robots themselves to the integration of the technology into existing fruit production systems. The potential benefits and challenges of these applied aspects will be a key focus of this research paper.

Overall, the autonomous fruit-harvesting robot system is a complex and multifaceted technology that holds great promise for the future of agriculture. By examining the various social, sustainable, technical, and applied aspects of this technology, I hope to shed light on its potential impacts and contributions to society and the environment.

\subsection{Social impacts, include distribution of wealth}
\subsection{Value = Sustainability}
\subsection{Existing solutions}
\subsection{Applied aspects (The target environment)}

\subsection{}

\section{Methods}
The goal of this research paper was to explore the social, sustainable, technical, and applied aspects of a fruit-harvesting robot system. To achieve this goal, I used a combination of literature research, brainstorming, and deductive reasoning.

Literature research was the primary method used to gather information for this study. I conducted a comprehensive review of the literature on robotics, including academic articles, textbooks, and news articles. 


I searched for sources using relevant keywords and databases, such as Google Scholar, Scopus, and IEEE Xplore. The literature review allowed us to identify the key issues and challenges related to the social, sustainable, technical, and applied aspects of fruit-harvesting robots.

In addition to literature research, I also used brainstorming and deductive reasoning to identify and explore additional questions and issues related to the topic. We held several brainstorming sessions to generate ideas and questions about the social, sustainable, technical, and applied aspects of fruit-harvesting robots. We then used deductive reasoning to narrow down the focus of the study and to develop the research questions and hypotheses.

Overall, the combination of literature research, brainstorming, and deductive reasoning allowed us to comprehensively explore the social, sustainable, technical, and applied aspects of a fruit-harvesting robot system and to identify the key issues and challenges related to this technology.

%TODO this is all trash

\section{Results}
What did we find out?


Equations should be numbered on the right side, such as:
\begin{align}
a_1& =b_1+c_1 \label{eq:a1} \\
a_2& =b_2+c_2-d_2+e_2 \label{eq:a2}
\end{align}
Equations are cross-referenced as \autoref{eq:a1} and \autoref{eq:a2}.

\section{Discussion}

\autoref{tab:extab} illustrates the use of tables.
It uses the \code{booktabs} package which provides improved typesetting of tables and \code{numprint} for the thousands separator.
\begin{table}[htbp]
  \caption{Sizes of compiler phases, lines of code.}\label{tab:extab}
  \centering
  \begin{tabular}{p{6cm}r} \toprule
      \emph{Compiler Phase} & \emph{Lines} \\
      \midrule
      FrontEnd & \numprint{92192} \\
      BackEnd & \numprint{29190} \\
      Code generation & \numprint{8957} \\
      \emph{Total size} & \emph{\numprint{130339}} \\
      \bottomrule
  \end{tabular}
\end{table}

\section{Additional meaningless text}
This section contains additional text to bring the example length to three pages



Lorem ipsum dolor sit amet, consectetur adipiscing elit. Quisque faucibus, arcu non malesuada condimentum, nibh est aliquet nunc, in tempus nisi urna id dolor. Suspendisse potenti. Sed nec accumsan massa, porttitor placerat purus. Mauris nibh tellus, lobortis ac posuere non, bibendum ut mi. Etiam et quam at arcu gravida ullamcorper. Donec eget tortor eros. Pellentesque habitant morbi tristique senectus et netus et malesuada fames ac turpis egestas. Donec quis odio tellus. Integer consequat vulputate dolor, sit amet sodales nibh dapibus eu. Quisque accumsan mauris tellus, ut sollicitudin sapien finibus in. Fusce congue, mauris vestibulum vulputate commodo, quam leo faucibus purus, id luctus augue est vitae dolor. Suspendisse id ultrices diam, eget viverra diam. Donec non sem mauris. Fusce sagittis neque justo, in mollis felis luctus in. Pellentesque tincidunt mauris a feugiat accumsan. Nulla eget sem nisi.

Cras enim tortor, luctus et vulputate vitae, condimentum quis massa. Nullam fermentum, lectus a mattis laoreet, eros nunc interdum nibh, in commodo justo ipsum quis mauris. Donec imperdiet faucibus lacinia. Phasellus malesuada porta arcu, nec molestie dui posuere quis. Donec porttitor, tellus id egestas feugiat, dui quam luctus dui, vel ornare metus lorem sit amet ex. Sed bibendum convallis condimentum. Vivamus eu consectetur felis. Sed turpis nisi, malesuada id augue eu, semper pulvinar metus. Nullam id ante sed mauris bibendum iaculis. Proin rhoncus justo mauris, vel iaculis nunc rhoncus in. Aenean nec lectus non eros mollis lacinia. Fusce at massa in nunc scelerisque egestas. Nulla in turpis ante. Quisque luctus at velit vitae iaculis.

Etiam nec sapien risus. Duis lorem felis, varius et purus at, malesuada pellentesque enim. Fusce lobortis ac elit eget feugiat. Nam purus libero, aliquam eu urna quis, volutpat eleifend leo. Ut a volutpat felis. Praesent lobortis sapien nunc, vel mattis nisi laoreet et. Praesent elementum ex a fringilla pulvinar. Vestibulum condimentum elementum pharetra. Mauris condimentum tempus risus, tincidunt viverra velit. Proin dictum ligula lectus, vitae euismod ligula rutrum non. Nunc non enim ultrices, sollicitudin nibh ut, lacinia ex. Nullam ac ullamcorper ante. Nullam tristique laoreet enim, sit amet suscipit risus imperdiet non. Vivamus id turpis egestas, viverra mauris non, sagittis purus.

Phasellus eget lobortis magna. Mauris faucibus elit eget magna gravida, nec ultrices eros consectetur. Quisque porttitor tincidunt nunc vitae gravida. Vestibulum laoreet tempus feugiat. Etiam sit amet molestie urna. Praesent libero nisi, sollicitudin accumsan ullamcorper auctor, mollis sed nulla. Pellentesque consequat, nibh ac ultrices pulvinar, tortor enim aliquet augue, non facilisis lorem arcu nec justo. Vestibulum posuere a metus nec aliquam. Quisque commodo, neque rhoncus scelerisque viverra, velit sapien mollis tellus, et scelerisque est massa vitae risus. Cras sollicitudin nisl sit amet suscipit lobortis. Vivamus sit amet arcu rhoncus, varius nisi id, viverra leo. In hac habitasse platea dictumst. Pellentesque placerat sem rutrum condimentum elementum. Cras euismod augue et luctus facilisis.

Phasellus feugiat vehicula dolor, eu dapibus ex aliquam vitae. Nunc tortor magna, lacinia id consectetur et, lacinia id enim. Nunc at consectetur odio, ac blandit ex. Aliquam pharetra mi vitae mauris maximus, in sagittis orci ultrices. Curabitur sagittis tortor sem, ornare aliquet dolor congue vitae. Etiam sed nunc ut leo gravida commodo. Maecenas rhoncus odio id tortor blandit dignissim. Proin sed tincidunt metus, eget pharetra odio. Nullam et imperdiet nisl. Praesent sagittis, nunc vel laoreet tristique, quam lorem varius quam, ut varius est purus sit amet tellus. Cras ac massa neque. Aliquam pulvinar auctor elit in tincidunt.

Maecenas iaculis odio at purus porta, ac tincidunt libero interdum. Aliquam sed sapien leo. Duis malesuada pharetra ex, eu vulputate est mattis at. Cras et lacus ac quam pellentesque efficitur et ut purus. Sed sed eros non justo gravida volutpat sed non libero. Proin imperdiet pretium mattis. Quisque sapien ex, dignissim vitae eleifend id, hendrerit at neque. Sed ultrices ante purus, nec hendrerit urna congue sed. Maecenas malesuada bibendum velit, convallis varius sapien volutpat vel. Ut malesuada pretium orci, eu elementum leo malesuada id. Etiam ullamcorper lobortis imperdiet. Phasellus ornare bibendum ante vitae vestibulum. Integer eu arcu sit amet ligula elementum blandit.

Praesent suscipit, purus eget tempor placerat, lectus lorem facilisis odio, at sagittis sapien leo ac justo. Etiam suscipit quis nisl vel posuere. Suspendisse vulputate arcu eu metus maximus, vitae scelerisque elit porta. Aliquam pulvinar libero in libero interdum iaculis. In metus ligula, vestibulum et ligula eu, sagittis mattis odio. Duis in convallis lorem. Nam at pharetra est. Donec eleifend fringilla odio vitae placerat. Integer sit amet eros eget odio condimentum placerat sit amet sed purus. Nulla at ultrices velit. Etiam ornare, purus non lacinia ullamcorper, nunc augue interdum magna, quis laoreet ante risus id sapien.

Aenean id leo gravida, sodales sem ut, imperdiet mauris. Proin pellentesque libero laoreet libero volutpat, quis euismod nibh commodo. Nam et dui vel augue convallis aliquam. Mauris magna diam, ultricies eget placerat porttitor, hendrerit non mi. Quisque pulvinar varius ex, sollicitudin vestibulum est cursus eu. Sed lobortis volutpat massa sed bibendum. Cras rutrum, sem sit amet iaculis egestas, nisi turpis vehicula ipsum, eget congue risus enim vehicula turpis. Curabitur sit amet tortor vitae augue tempor consequat.

\section{Bibliographic References}
The bibliographic reference list are shown at the end of the paper;
starting with an unnumbered heading \emph{``References''}. The list of
references should be sorted in alphabetic order according to the first
author's surname.
The first author's name is printed surname first and subsequent authors are printed with the first name first.

Citations are stated within the body text using the name of the
reference enclosed within parentheses, e.g., \cite{Pantelides:1988}. If
more than one reference is cited at the same place, the list should be
sorted, separated by semicolons and within parentheses, e.g.,
\cite{DuffReid:1978,Pierce:2002,Plotkin:1981}.
It is also possible use the textcite command to include a reference more naturally in the text: \textcite{Pantelides:1988} wrote something interesting in his paper.
If there is a DOI for the reference, use it instead of URLs in the bibliography.

Citations for relevant Modelica language specifications (MLSs) are provided as
MLSv32r2 \cite{MLSv32r2}, MLSv33r1 \cite{MLSv33r1}, and MLSv34 \cite{MLSv34}.

All entries in the reference list should be cited in the manuscript.

In order to populate the bibliography with different kinds of entries to show how they should be printed, here are some additional citations:

\begin{itemize}
\item A book \cite{Kernighan:1988}.
\item A conference paper \cite{colaco:2003}.
\item A couple of fake PhD, MSc, and BSc thesis \cite{Doe:PhD,Doe:MSc,Doe:BSc}.
\end{itemize}

\section*{Acknowledgements}

This work has been supported by X (grant numbers aaaa, bbbb); Y (grant number cccc) and Z (grant number dddd).
The authors would also like to thank A and B for their support with implementation of the software.

\printbibliography

\newpage
\onecolumn
\appendix
\section[Section 1]{Assignments for section 1}
\label{paper-w1}
This appendix includes initial definitions of the fruit picker robot including diagrams and cartoons. In addition it features discussions on legal and ethical aspects, as well as a discussion on human-automation interaction issues.
\includepdf[pages=-,openright,width=\textwidth]{resources/W1.pdf}
\section[Section 1]{Assignments for section 2}
\label{paper-w2}
This appendix revolves around motivations or key drivers for the system from multiple perspectives. This discussion includes topics such as cost, labor dependance, product quality, sustainability, and a look at UNs SDGs. It also includes a safety and security discussion followed by a HAZID analysis. Finally a business model canvas is included.
\includepdf[pages=-,openright,width=\textwidth]{resources/W2.pdf}
\section[Section 1]{Assignments for section 3}
\label{paper-w3}
This appendix contains technical information on the organization of software in the system. It includes a system diagram, a layered architecture diagram, a reflection on data collection, and a reflection about learning in the system. 
\includepdf[pages=-,openright,width=\textwidth]{resources/W3.pdf}
\section[Section 1]{Assignments for section 4}
\label{paper-w4}
This final appendix describes capabilities of autonomous systems, different frameworks used for robotics, industrial applications, operational boundaries of systems, and finally an overview of overall capabilities required of the fruit-picker system.
\includepdf[pages=-,openright,width=\textwidth]{resources/W4.pdf}





%\appendix
%\renewcommand{\appendixname}{W1} %% So we get 'Paper X' displayed instead
%
%
%\chapter[Section 1]{Assignments for section 1}
%\label{paper-w1}
%
%
%\includepdf[pages=-,openright,width=\textwidth]{resources/W1.pdf}

%
%You can omit the \texttt{.pdf} when specifying the \texttt{Filename}. Also you should include always include the option \texttt{openright} since it would look strange to have the paper starting at the back of the cover page.
%
%There are more options like only adding specific pages:
%\begin{verbatim}
%\includepdf{pages=2-6,openright}{Filename.pdf}
%\end{verbatim}

%For more options see Appendix~\ref{paper-b} where the most important pages of the \texttt{pdfpages} manual were inlcuded using \texttt{pdfpages}.


%%% Command to include a PDF file directly including all pages:


%\chapter[Short Title of Paper B]{Title of Paper B}
%\label{paper-b}
%Short descriptive text of paper follows here.
%
%Here we included the first five pages of the \texttt{pdfpages} manual itself.
%
%\includepdf[pages=1-5,openright]{fig/pdfpages}
%

\end{document}
